\documentclass[a4paper]{article}
\usepackage[utf8]{inputenc}
\usepackage{verbatim}
\usepackage{german}
\usepackage{graphicx}
\usepackage{fancyhdr}
\usepackage{alltt}
\pagestyle{fancy}
\fancyhead[R]{Mario Groneick, Marcel Brockskothen}
\fancyfoot[R]{\includegraphics[width=3cm]{logo_hs_bochum}}
\author{Mario Groneick, Marcel Brockskothen}
\begin{document}
\begin{titlepage}
\vskip 200pt
%\maketitle
\begin{center}
\includegraphics{BO-Logo_o_Wortmarke.png}
\vskip 20pt
\rule{10cm}{0.7pt}
\vskip 20pt
\huge{Ecl-Emma-Jacoco}
\rule{10cm}{0.7pt}
\vskip 20pt
\large{Mario Groneick, Marcel Brockskothen}
\vskip 20pt
\large{Kurs: Softwareschnittstellen und Softwarequalität}
\vskip 20pt
\large{Betreuer: Prof. Dr. Ursula Oesing}
\vskip 100pt
\large{24.7.2018}
\end{center}
\end{titlepage}

\tableofcontents
\section{Ecl-Emma-Jacoco}
Ecl-Emma ist ein Eclipse Plugin um die Testbdeckung in Eclipse zu messen.
Seit Version 2.0 basiert Ecl-Emma auf der Jacoco Testabdeckungs Bibliothek für Java.
Davor basierste es auf der Emma Bibliotheck von Vlad Roubtsov. Das Ecl-Emma-Plugin unterstützt
sowohl Junit als auch TestNG als Testframework.
\subsection{Installation von Ecl-Emma-Jacoco}
\subsection{Nutzung von Ecl-Emma-Jacoco}
\subsection{Theoretische Testabdeckung und Ecl-Emma}
\section{Bistro-Verwaltungs-App}
\section{Testabdeckung der Bistro-Verwaltungs-App}
\section{Fazit}
\begin{thebibliography}{1}
	\bibitem{hoff} \emph{Software Qualität, Dirk Hoffman} (2006)
	\bibitem{wiki} \emph{https://de.wikipedia.org/wiki/Kontrollflussorientierte\_Testverfahren} Zugriff 16.6.2018
	\bibitem{ecl-emma} \emph{https://www.jacoco.org/index.html} Zugriff 16.6.2018
	\bibitem{jacoco} \emph{https://www.jacoco.org/jacoco/trunk/doc/counters.html} Zugriff 16.6.2018
	\bibitem{code2flow} \emph{https://code2flow.com/} Zugriff 16.6.2018
\end{thebibliography}
\end{document}
